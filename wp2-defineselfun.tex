\begin{workpackage}{Definition of the selection function}
  \label{wp:selfundefinition}
  \wpstart{1} %Starting Month
  \wpend{12} %End Month
  \wptype{RTD} %RTD, DEM, MGT, or OTHER
  \wplead{MPG}
  \personmonths{ULEI}{1}
  \personmonths*{MPIA}{2}
  \personmonths{INAF}{1}
  \personmonths{UCAM}{1}
  \personmonths{NYU}{1}
  \personmonths{MONA}{1}
  
  \makewptable % Work package summary table

  % Work Package Objectives
  \begin{wpobjectives}
    This objective of this work package is to research and implement a precise mathematical formulation of the concept of a survey selection function. The results will be written up as a scientific paper (to be published in the open access peer-reviewed literature) that will guide the rest of the work to be done within {\acro}.  The mathematical formulation of the selection function will account for the following aspects:
    \begin{itemize}
        \item Applicability to arbitrary astronomical sky surveys.
        \item Selection functions for combinations of surveys can be described within the same formalism.
        \item Accommodate multiple selection layers (e.g., the selection function intrinsic to a survey combined with additional selections imposed by the user of the data).
        \item Unknowable aspects of the selection function should be included. For example, it is not guaranteed that all sources of data loss along the detection-telemetry-processing chain for Gaia can be traced, but this can be approximated through knowing which repeated measurements were included in the catalogue when expected.
    \end{itemize}
  \end{wpobjectives}

  % Work Package Description
  \begin{wpdescription}
    % Divide work package into multiple tasks.
    % Use \wptask command
    % syntax: \wptask{leader}{contributors}{start-m}{end-m}{title}{description}   

    \wptask{MPIA}{MPIA}{1}{12}{Work package management}{
      \label{task:wp2coordination}
      Management and coordination of the specific work on defining the selection function and coordination of writing the paper corresponding to deliverable \ref{dev:wp2document}. This includes planning the work to be done, assigning sub-tasks and organizing the necessary meetings to discuss progress.
      
      \textsf{1 MPIA \pem}
    }
    \wptask{MPIA}{All other}{1}{12}{R\&D formulation selection function}{
      \label{task:wp2rtd}
      Research and implement the mathematical formulation of survey selection functions.
      
      \textsf{1 MPIA + 1 ULEI + 1 INAF + 1 UCAM + 1 MONA + 1 NYU \pems}
    }

    \paragraph{Role of partners}
    \begin{description}
      \item[MPIA] will lead Task~\ref{task:wp2coordination}.
      \item[ULEI, INAF, UCAM, NYU, MONA] will contribute to the research and implementation of the survey selection function mathematical formulation and will contribute to writing the paper corresponding to deliverable \ref{dev:wp2document}.
    \end{description}
  \end{wpdescription}

  % Work Package Deliverable
  \begin{wpdeliverables}
    % \wpdeliverable[date]{R}{PU}{A report on \ldots}
    \wpdeliverable[12]{MPIA}{R}{PU}{Document on the mathematical formulation of survey selection functions (to be submitted to open access peer-reviewed journal).}\label{dev:wp2document}
  \end{wpdeliverables}

\end{workpackage}


%%% Local Variables:
%%% mode: latex
%%% TeX-master: "proposal-main"
%%% End:
