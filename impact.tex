\chapter{Impact}
\label{cha:impact}


\section{Expected impacts} 
\label{sec:expected-impact}
\instructions{
\begin{itemize}
    \item Describe how your project will contribute to:
    \begin{itemize}
        \item each of the expected impacts mentioned in the work programme, under the relevant topic;
        \item any substantial impacts not mentioned in the work programme, that would enhance innovation capacity; create new market opportunities, strengthen competitiveness and growth of companies, address issues related to climate change or the environment, or  bring other important benefits for society
    \end{itemize}
    \item Describe any barriers/obstacles, and any framework conditions (such as regulation, standards, public acceptance, workforce considerations, financing of follow-up steps, cooperation of other links in the value chain), that may determine whether and to what extent the expected impacts will be achieved. (This should not include any risk factors concerning implementation, as covered in section 3.2.)
\end{itemize}
}

We expect the following impacts directly related to the work programme:
\begin{itemize}
    \item The availability of a detailed Gaia selection function coupled with the data and tools to use it will \textbf{enhance the quality and reproducibility the scientific exploitation of data from a flagship European space mission}. Thus great value is added to the existing European efforts dedicated to the data processing for the Gaia mission as well as to the efforts to produce the publicly available Gaia data releases.
    \item A readily available Gaia selection function will enable science cases to be addressed that are currently hampered by the lack of a detailed description of Gaia survey selection biases. This will translate to \textbf{an increase in the number of scientific publications based on European space missions}.
    %\item We will make data products and tools available for a much more advanced analysis of Gaia and other survey data than is currently possible.
    \item Although the participants on this proposal have all worked together in various constellations in the past on different topics, this is the first time that this group of experts, inspired and excited to invest time in the details of the Gaia selection function, will work closely together, thus enhancing an existing collaboration and broadening the expertise of the partners involved.
    \item The partner institutes will benefit from having Gaia and survey selection function experts in house, expertise which can then be turned into higher quality scientific exploitation of astronomical data and more publications (based on data from European space missions and surveys).
    \item \textbf{The European scientific community will gain a competitive advantage} in the exploitation of space mission data with respect to their peers worldwide through the expertise and the tools built within {\acro}. The expertise will be transferred to the community through the workshops that {\acro} will organize, and through presentations at relevant scientific conferences.
    \item The expertise on the Gaia selection function will be transferred to DPAC. This will facilitate turning the selection function into a standard DPAC data product for the community. In addition a detailed selection function allows for a more careful validation of the DPAC data products. For example, features in the population statistics of sub-samples of the Gaia data can then be correctly attributed to a selection bias or a problem in the data processing. The result will be better quality control before each Gaia data release.
    \item The expertise gained in this project can be transferred to other surveys. This is already partly planned in this proposal through work package \ref{wp:selfuncombine} on combined selection functions. However we expect that other survey teams will seek out the expertise developed in this project and  pick up the methods and tools we make available. This will enhance both European and international activities. \textbf{Examples of future European projects that stand to benefit are the 4MOST and WEAVE spectroscopic surveys and the Euclid and Plato space missions.} Examples of \textbf{international projects that will benefit are future surveys such as LSST and SDSS-V (USA) as well as existing surveys such as GALAH (Australia)} which rely heavily on complementary Gaia data.
    \item \textbf{The scientific community world-wide will benefit from the higher quality scientific analyses} that are possible with the public availability of the Gaia survey selection function.
\end{itemize}

On a broader societal level the impact will be through the expertise built up with understanding in detail how the interpretation of large amounts of collected data is affected by selection biases and how this limitation can be addressed through a proper description of the way the data was collected. This will be of benefit to any data science application and we expect that some of the researchers funded through this proposal might opt for a future career in European companies of which the business contains a large data science component.

\paragraph{Potential barriers/obstacles} Other than the normal risks associated with scientific research we do not foresee any barriers to achieving the impacts above. The work proposed is not subject to regulation standards and presents no workforce issues. Although follow-up steps could be foreseen, e.g.\ further detailing the Gaia selection function beyond what is achieved in {\acro}, the absence of such steps will not present a problem. {\acro} is self-contained in that respect. The project implementation does of course have some risk factors as listed in \tabref{table:risks}.

\section{Measures to maximize impact} 
\label{sec:maximize-impact}

\subsection{Dissemination and exploitation of results}
\label{sec:dissemination-exploitation}
\instructions{
\begin{itemize}
    \item Provide a draft `plan for the dissemination and exploitation of the project's results'. Please note that such a draft plan is an admissibility condition, unless the work programme topic explicitly states that such a plan is not required.\\
    Show how the proposed measures will help to achieve the expected impact of the project.\\
    The plan, should be proportionate to the scale of the project, and should contain measures to be implemented both during and after the end of the project. For innovation actions, in particular, please describe a credible path to deliver these innovations to the market.
    \item Include a business plan where relevant.
    \item As relevant, include information on how the participants will manage the research data generated and/or collected during the project, in particular addressing the following issues:
    \begin{itemize}
        \item What types of data will the project generate/collect? o What standards will be used? 
        \item How will this data be exploited and/or shared/made accessible for verification and re-use? If data cannot be made available, explain why.
        \item How will this data be curated and preserved? 
        \item How will the costs for data curation and preservation be covered?
    \end{itemize}
    \item Outline the strategy for knowledge management and protection. Include measures to provide open access (free on-line access, as the `green' or `gold' model) to peer-reviewed scientific publications which might result from the project.
\end{itemize}
}

The area in which we expect to have the highest impact is the scientific exploitation of results from the Gaia mission and other surveys. Correspondingly our dissemination efforts will be aimed at scientists using astronomical survey data and we will primarily use the standard academic channels:
\begin{itemize}
    \item Presentation, description and documentation of our results through scientific publications in peer-reviewed journals. We will make use of open access journals (`gold' open access) and we will always make our publications available through \url{https://arxiv.org} (`green' open access).
    \item Our results will be presented and promoted at scientific meetings where we aim to reach a diverse audience through the variety of science applications addressed as part of our efforts.
    \item We will organize three community workshops in which the Gaia selection function experts will offer training to interested scientists in the use of our data products and tools (deliverables \ref{dev:wp1midtterm}, \ref{dev:wp1community} \ref{dev:wp1closing}). Our results from the science applications in work package \ref{wp:scienceappl} will serve as excellent worked examples.
    %\item We will discuss with colleagues providing courses in data analysis and statistics to science students how our methods and tools may be used in such courses. We will provide support to adapt our tools to the needs of the course instructors. \memo{Make visibile in WPs or drop.}
\end{itemize}

This project will collect the data needed to understand the Gaia selection function, such as the details of the actual scan law used or of the spacecraft operation interruptions. These data will be part of the tools needed to use the Gaia selection function and thus will be stored in a form suitable for computer use or for storage in a database.
\begin{itemize}
    \item All data generated/collected in this project will be made available publicly. This will be done through the ESA Gaia archives and any other astronomical data centres interested in hosting these data. In addition the data will be deposited with \url{https://zenodo.org}.
    \item The source code for the computer tools will be made available publicly through well known open access source code repositories such as \url{http://github.com} or through code hosting services available at the institutes participating in this proposal.
    \item A website will be setup (hosted by Leiden University) which will serve as entry point to the data, tools, and documentation generated in this project. To enhance visibility cross-links will be made with, for example, ESA websites related to Gaia.
\end{itemize}

The Gaia mission operations (including mission extensions) will end at the latest around end 2024 and the data processing will continue for a number years thereafter before the final (legacy) Gaia archive version can be released. This is beyond the lifetime of the project proposed here and we thus intend to transfer the expertise on the selection function to the DPAC aiming to make the selection function a data product provided by DPAC. This is most naturally provided together with the documentation through the ESA archive. Hence a long term curation of the corresponding data is foreseen as part of the Gaia legacy archive maintained by ESA. The Gaia archive is expected to remain the standard in astrometric and photometric data for decades to come. 

The long term curation of the {\acro} tools and corresponding source code is best addressed by embedding it in an existing eco-system. The Astropy project\footnote{https://www.astropy.org/} is a good candidate; as stated on the website, `Astropy is a community effort to develop a common core package for Astronomy in Python and foster an ecosystem of inter-operable astronomy packages'. One of the {\acro} team (Price-Whelan) is a major contributor to Astropy \cite{astropy2018} and with his contacts and expertise in hand we will work to ensure the {\acro} tools are embedded in Astropy or released as an affiliated package.

\subsection{Communication activities}
\label{sec:communication-activities}
\instructions{
\begin{itemize}
    \item Describe the proposed communication measures for promoting the project and its findings during the period of the grant. Measures should be proportionate to the scale of the project, with clear objectives.  They should be tailored to the needs of different target audiences, including groups beyond the project's own community.
\end{itemize}
}

As mentioned above the community explicitly targeted by this proposal, professional astronomers, will be reached through the well known academic outreach channels: scientific publications, presentations at scientific conferences, and open access to the data and tools resulting from this project. In addition we plan to engage the general public through media releases of the results from application of the Gaia selection function to specific science cases, highlighting how the work on the selection function enables much better interpretation of, for example, binary star and exoplanet population statistics. This will also be used as a hook to explain how a science like astronomy can enhance data analysis capabilities for society at large. After all, the concept of a `selection function' is applicable to population modelling across all fields of research; spelling out a framework that is both intuitive and mathematically sound matters widely, far beyond Gaia or astrophysics.
